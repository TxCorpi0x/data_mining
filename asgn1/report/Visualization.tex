%% bare_jrnl.tex
%% V1.4b
%% 2015/08/26
%% by Michael Shell
%% see http://www.michaelshell.org/
%% for current contact information.
%%
%% This is a skeleton file demonstrating the use of IEEEtran.cls
%% (requires IEEEtran.cls version 1.8b or later) with an IEEE
%% journal paper.
%%
%% Support sites:
%% http://www.michaelshell.org/tex/ieeetran/
%% http://www.ctan.org/pkg/ieeetran
%% and
%% http://www.ieee.org/

%%*************************************************************************
%% Legal Notice:
%% This code is offered as-is without any warranty either expressed or
%% implied; without even the implied warranty of MERCHANTABILITY or
%% FITNESS FOR A PARTICULAR PURPOSE! 
%% User assumes all risk.
%% In no event shall the IEEE or any contributor to this code be liable for
%% any damages or losses, including, but not limited to, incidental,
%% consequential, or any other damages, resulting from the use or misuse
%% of any information contained here.
%%
%% All comments are the opinions of their respective authors and are not
%% necessarily endorsed by the IEEE.
%%
%% This work is distributed under the LaTeX Project Public License (LPPL)
%% ( http://www.latex-project.org/ ) version 1.3, and may be freely used,
%% distributed and modified. A copy of the LPPL, version 1.3, is included
%% in the base LaTeX documentation of all distributions of LaTeX released
%% 2003/12/01 or later.
%% Retain all contribution notices and credits.
%% ** Modified files should be clearly indicated as such, including  **
%% ** renaming them and changing author support contact information. **
%%*************************************************************************


% *** Authors should verify (and, if needed, correct) their LaTeX system  ***
% *** with the testflow diagnostic prior to trusting their LaTeX platform ***
% *** with production work. The IEEE's font choices and paper sizes can   ***
% *** trigger bugs that do not appear when using other class files.       ***                          ***
% The testflow support page is at:
% http://www.michaelshell.org/tex/testflow/



\documentclass[journal]{IEEEtran}
%
% If IEEEtran.cls has not been installed into the LaTeX system files,
% manually specify the path to it like:
% \documentclass[journal]{../sty/IEEEtran}

\usepackage[pdftex]{graphicx}
\usepackage{hyperref}
\graphicspath{{../pdf/}{../jpeg/}}
\DeclareGraphicsExtensions{.pdf,.jpeg,.png}




% Some very useful LaTeX packages include:
% (uncomment the ones you want to load)


% *** MISC UTILITY PACKAGES ***
%
%\usepackage{ifpdf}
% Heiko Oberdiek's ifpdf.sty is very useful if you need conditional
% compilation based on whether the output is pdf or dvi.
% usage:
% \ifpdf
%   % pdf code
% \else
%   % dvi code
% \fi
% The latest version of ifpdf.sty can be obtained from:
% http://www.ctan.org/pkg/ifpdf
% Also, note that IEEEtran.cls V1.7 and later provides a builtin
% \ifCLASSINFOpdf conditional that works the same way.
% When switching from latex to pdflatex and vice-versa, the compiler may
% have to be run twice to clear warning/error messages.






% *** CITATION PACKAGES ***
%
%\usepackage{cite}
% cite.sty was written by Donald Arseneau
% V1.6 and later of IEEEtran pre-defines the format of the cite.sty package
% \cite{} output to follow that of the IEEE. Loading the cite package will
% result in citation numbers being automatically sorted and properly
% "compressed/ranged". e.g., [1], [9], [2], [7], [5], [6] without using
% cite.sty will become [1], [2], [5]--[7], [9] using cite.sty. cite.sty's
% \cite will automatically add leading space, if needed. Use cite.sty's
% noadjust option (cite.sty V3.8 and later) if you want to turn this off
% such as if a citation ever needs to be enclosed in parenthesis.
% cite.sty is already installed on most LaTeX systems. Be sure and use
% version 5.0 (2009-03-20) and later if using hyperref.sty.
% The latest version can be obtained at:
% http://www.ctan.org/pkg/cite
% The documentation is contained in the cite.sty file itself.






% *** GRAPHICS RELATED PACKAGES ***
%
\ifCLASSINFOpdf
  % \usepackage[pdftex]{graphicx}
  % declare the path(s) where your graphic files are
  % \graphicspath{{../pdf/}{../jpeg/}}
  % and their extensions so you won't have to specify these with
  % every instance of \includegraphics
  % \DeclareGraphicsExtensions{.pdf,.jpeg,.png}
\else
  % or other class option (dvipsone, dvipdf, if not using dvips). graphicx
  % will default to the driver specified in the system graphics.cfg if no
  % driver is specified.
  % \usepackage[dvips]{graphicx}
  % declare the path(s) where your graphic files are
  % \graphicspath{{../eps/}}
  % and their extensions so you won't have to specify these with
  % every instance of \includegraphics
  % \DeclareGraphicsExtensions{.eps}
\fi
% graphicx was written by David Carlisle and Sebastian Rahtz. It is
% required if you want graphics, photos, etc. graphicx.sty is already
% installed on most LaTeX systems. The latest version and documentation
% can be obtained at: 
% http://www.ctan.org/pkg/graphicx
% Another good source of documentation is "Using Imported Graphics in
% LaTeX2e" by Keith Reckdahl which can be found at:
% http://www.ctan.org/pkg/epslatex
%
% latex, and pdflatex in dvi mode, support graphics in encapsulated
% postscript (.eps) format. pdflatex in pdf mode supports graphics
% in .pdf, .jpeg, .png and .mps (metapost) formats. Users should ensure
% that all non-photo figures use a vector format (.eps, .pdf, .mps) and
% not a bitmapped formats (.jpeg, .png). The IEEE frowns on bitmapped formats
% which can result in "jaggedy"/blurry rendering of lines and letters as
% well as large increases in file sizes.
%
% You can find documentation about the pdfTeX application at:
% http://www.tug.org/applications/pdftex





% *** MATH PACKAGES ***
%
%\usepackage{amsmath}
% A popular package from the American Mathematical Society that provides
% many useful and powerful commands for dealing with mathematics.
%
% Note that the amsmath package sets \interdisplaylinepenalty to 10000
% thus preventing page breaks from occurring within multiline equations. Use:
%\interdisplaylinepenalty=2500
% after loading amsmath to restore such page breaks as IEEEtran.cls normally
% does. amsmath.sty is already installed on most LaTeX systems. The latest
% version and documentation can be obtained at:
% http://www.ctan.org/pkg/amsmath





% *** SPECIALIZED LIST PACKAGES ***
%
%\usepackage{algorithmic}
% algorithmic.sty was written by Peter Williams and Rogerio Brito.
% This package provides an algorithmic environment fo describing algorithms.
% You can use the algorithmic environment in-text or within a figure
% environment to provide for a floating algorithm. Do NOT use the algorithm
% floating environment provided by algorithm.sty (by the same authors) or
% algorithm2e.sty (by Christophe Fiorio) as the IEEE does not use dedicated
% algorithm float types and packages that provide these will not provide
% correct IEEE style captions. The latest version and documentation of
% algorithmic.sty can be obtained at:
% http://www.ctan.org/pkg/algorithms
% Also of interest may be the (relatively newer and more customizable)
% algorithmicx.sty package by Szasz Janos:
% http://www.ctan.org/pkg/algorithmicx




% *** ALIGNMENT PACKAGES ***
%
%\usepackage{array}
% Frank Mittelbach's and David Carlisle's array.sty patches and improves
% the standard LaTeX2e array and tabular environments to provide better
% appearance and additional user controls. As the default LaTeX2e table
% generation code is lacking to the point of almost being broken with
% respect to the quality of the end results, all users are strongly
% advised to use an enhanced (at the very least that provided by array.sty)
% set of table tools. array.sty is already installed on most systems. The
% latest version and documentation can be obtained at:
% http://www.ctan.org/pkg/array


% IEEEtran contains the IEEEeqnarray family of commands that can be used to
% generate multiline equations as well as matrices, tables, etc., of high
% quality.




% *** SUBFIGURE PACKAGES ***
%\ifCLASSOPTIONcompsoc
%  \usepackage[caption=false,font=normalsize,labelfont=sf,textfont=sf]{subfig}
%\else
%  \usepackage[caption=false,font=footnotesize]{subfig}
%\fi
% subfig.sty, written by Steven Douglas Cochran, is the modern replacement
% for subfigure.sty, the latter of which is no longer maintained and is
% incompatible with some LaTeX packages including fixltx2e. However,
% subfig.sty requires and automatically loads Axel Sommerfeldt's caption.sty
% which will override IEEEtran.cls' handling of captions and this will result
% in non-IEEE style figure/table captions. To prevent this problem, be sure
% and invoke subfig.sty's "caption=false" package option (available since
% subfig.sty version 1.3, 2005/06/28) as this is will preserve IEEEtran.cls
% handling of captions.
% Note that the Computer Society format requires a larger sans serif font
% than the serif footnote size font used in traditional IEEE formatting
% and thus the need to invoke different subfig.sty package options depending
% on whether compsoc mode has been enabled.
%
% The latest version and documentation of subfig.sty can be obtained at:
% http://www.ctan.org/pkg/subfig




% *** FLOAT PACKAGES ***
%
%\usepackage{fixltx2e}
% fixltx2e, the successor to the earlier fix2col.sty, was written by
% Frank Mittelbach and David Carlisle. This package corrects a few problems
% in the LaTeX2e kernel, the most notable of which is that in current
% LaTeX2e releases, the ordering of single and double column floats is not
% guaranteed to be preserved. Thus, an unpatched LaTeX2e can allow a
% single column figure to be placed prior to an earlier double column
% figure.
% Be aware that LaTeX2e kernels dated 2015 and later have fixltx2e.sty's
% corrections already built into the system in which case a warning will
% be issued if an attempt is made to load fixltx2e.sty as it is no longer
% needed.
% The latest version and documentation can be found at:
% http://www.ctan.org/pkg/fixltx2e


%\usepackage{stfloats}
% stfloats.sty was written by Sigitas Tolusis. This package gives LaTeX2e
% the ability to do double column floats at the bottom of the page as well
% as the top. (e.g., "\begin{figure*}[!b]" is not normally possible in
% LaTeX2e). It also provides a command:
%\fnbelowfloat
% to enable the placement of footnotes below bottom floats (the standard
% LaTeX2e kernel puts them above bottom floats). This is an invasive package
% which rewrites many portions of the LaTeX2e float routines. It may not work
% with other packages that modify the LaTeX2e float routines. The latest
% version and documentation can be obtained at:
% http://www.ctan.org/pkg/stfloats
% Do not use the stfloats baselinefloat ability as the IEEE does not allow
% \baselineskip to stretch. Authors submitting work to the IEEE should note
% that the IEEE rarely uses double column equations and that authors should try
% to avoid such use. Do not be tempted to use the cuted.sty or midfloat.sty
% packages (also by Sigitas Tolusis) as the IEEE does not format its papers in
% such ways.
% Do not attempt to use stfloats with fixltx2e as they are incompatible.
% Instead, use Morten Hogholm'a dblfloatfix which combines the features
% of both fixltx2e and stfloats:
%
% \usepackage{dblfloatfix}
% The latest version can be found at:
% http://www.ctan.org/pkg/dblfloatfix




%\ifCLASSOPTIONcaptionsoff
%  \usepackage[nomarkers]{endfloat}
% \let\MYoriglatexcaption\caption
% \renewcommand{\caption}[2][\relax]{\MYoriglatexcaption[#2]{#2}}
%\fi
% endfloat.sty was written by James Darrell McCauley, Jeff Goldberg and 
% Axel Sommerfeldt. This package may be useful when used in conjunction with 
% IEEEtran.cls'  captionsoff option. Some IEEE journals/societies require that
% submissions have lists of figures/tables at the end of the paper and that
% figures/tables without any captions are placed on a page by themselves at
% the end of the document. If needed, the draftcls IEEEtran class option or
% \CLASSINPUTbaselinestretch interface can be used to increase the line
% spacing as well. Be sure and use the nomarkers option of endfloat to
% prevent endfloat from "marking" where the figures would have been placed
% in the text. The two hack lines of code above are a slight modification of
% that suggested by in the endfloat docs (section 8.4.1) to ensure that
% the full captions always appear in the list of figures/tables - even if
% the user used the short optional argument of \caption[]{}.
% IEEE papers do not typically make use of \caption[]'s optional argument,
% so this should not be an issue. A similar trick can be used to disable
% captions of packages such as subfig.sty that lack options to turn off
% the subcaptions:
% For subfig.sty:
% \let\MYorigsubfloat\subfloat
% \renewcommand{\subfloat}[2][\relax]{\MYorigsubfloat[]{#2}}
% However, the above trick will not work if both optional arguments of
% the \subfloat command are used. Furthermore, there needs to be a
% description of each subfigure *somewhere* and endfloat does not add
% subfigure captions to its list of figures. Thus, the best approach is to
% avoid the use of subfigure captions (many IEEE journals avoid them anyway)
% and instead reference/explain all the subfigures within the main caption.
% The latest version of endfloat.sty and its documentation can obtained at:
% http://www.ctan.org/pkg/endfloat
%
% The IEEEtran \ifCLASSOPTIONcaptionsoff conditional can also be used
% later in the document, say, to conditionally put the References on a 
% page by themselves.




% *** PDF, URL AND HYPERLINK PACKAGES ***
%
%\usepackage{url}
% url.sty was written by Donald Arseneau. It provides better support for
% handling and breaking URLs. url.sty is already installed on most LaTeX
% systems. The latest version and documentation can be obtained at:
% http://www.ctan.org/pkg/url
% Basically, \url{my_url_here}.




% *** Do not adjust lengths that control margins, column widths, etc. ***
% *** Do not use packages that alter fonts (such as pslatex).         ***
% There should be no need to do such things with IEEEtran.cls V1.6 and later.
% (Unless specifically asked to do so by the journal or conference you plan
% to submit to, of course. )


% correct bad hyphenation here
\hyphenation{op-tical net-works semi-conduc-tor}


\begin{document}
\bstctlcite{IEEEexample:BSTcontrol}
%
% paper title
% Titles are generally capitalized except for words such as a, an, and, as,
% at, but, by, for, in, nor, of, on, or, the, to and up, which are usually
% not capitalized unless they are the first or last word of the title.
% Linebreaks \\ can be used within to get better formatting as desired.
% Do not put math or special symbols in the title.
\title{Visualization Plots Strength and Weakness}
%
%
% author names and IEEE memberships
% note positions of commas and nonbreaking spaces ( ~ ) LaTeX will not break
% a structure at a ~ so this keeps an author's name from being broken across
% two lines.
% use \thanks{} to gain access to the first footnote area
% a separate \thanks must be used for each paragraph as LaTeX2e's \thanks
% was not built to handle multiple paragraphs
%

\author{MEHDI VALINEJAD,~\IEEEmembership{Student}}% <-this % stops a space
% \thanks{M. Shell was with the Department
% of Electrical and Computer Engineering, Georgia Institute of Technology, Atlanta,
% GA, 30332 USA e-mail: (see http://www.michaelshell.org/contact.html).}% <-this % stops a space
% \thanks{J. Doe and J. Doe are with Anonymous University.}% <-this % stops a space
% \thanks{Manuscript received April 19, 2005; revised August 26, 2015.}



% note the % following the last \IEEEmembership and also \thanks - 
% these prevent an unwanted space from occurring between the last author name
% and the end of the author line. i.e., if you had this:
% 
% \author{....lastname \thanks{...} \thanks{...} }
%                     ^------------^------------^----Do not want these spaces!
%
% a space would be appended to the last name and could cause every name on that
% line to be shifted left slightly. This is one of those "LaTeX things". For
% instance, "\textbf{A} \textbf{B}" will typeset as "A B" not "AB". To get
% "AB" then you have to do: "\textbf{A}\textbf{B}"
% \thanks is no different in this regard, so shield the last } of each \thanks
% that ends a line with a % and do not let a space in before the next \thanks.
% Spaces after \IEEEmembership other than the last one are OK (and needed) as
% you are supposed to have spaces between the names. For what it is worth,
% this is a minor point as most people would not even notice if the said evil
% space somehow managed to creep in.



% The paper headers
\markboth{Visualization Plots Strength and Weakness, MARCH~2024}%
{Shell \MakeLowercase{\textit{et al.}}: Visualization Plots Strength and Weakness}
% The only time the second header will appear is for the odd numbered pages
% after the title page when using the twoside option.
% 
% *** Note that you probably will NOT want to include the author's ***
% *** name in the headers of peer review papers.                   ***
% You can use \ifCLASSOPTIONpeerreview for conditional compilation here if
% you desire.




% If you want to put a publisher's ID mark on the page you can do it like
% this:
%\IEEEpubid{0000--0000/00\$00.00~\copyright~2015 IEEE}
% Remember, if you use this you must call \IEEEpubidadjcol in the second
% column for its text to clear the IEEEpubid mark.



% use for special paper notices
%\IEEEspecialpapernotice{(Invited Paper)}




% make the title area
\maketitle

% As a general rule, do not put math, special symbols or citations
% in the abstract or keywords.
\begin{abstract}
Current report states the strength and weakness aspects of box, histogram, scatter and violin plots. the complete source code is available at
\url{https://github.com/TxCorpi0x/data_mining} under asgn1 directory.
\end{abstract}

\begin{IEEEkeywords}
box plot, histogram plot, scatter plot, violin plot, weakness, strength.
\end{IEEEkeywords}






% For peer review papers, you can put extra information on the cover
% page as needed:
% \ifCLASSOPTIONpeerreview
% \begin{center} \bfseries EDICS Category: 3-BBND \end{center}
% \fi
%
% For peerreview papers, this IEEEtran command inserts a page break and
% creates the second title. It will be ignored for other modes.
\IEEEpeerreviewmaketitle



% The very first letter is a 2 line initial drop letter followed
% by the rest of the first word in caps.
% 
% form to use if the first word consists of a single letter:
% \IEEEPARstart{A}{demo} file is ....
% 
% form to use if you need the single drop letter followed by
% normal text (unknown if ever used by the IEEE):
% \IEEEPARstart{A}{}demo file is ....
% 
% Some journals put the first two words in caps:
% \IEEEPARstart{T}{his demo} file is ....
% 
% Here we have the typical use of a "T" for an initial drop letter
% and "HIS" in caps to complete the first word.
\section{Introduction}

\IEEEPARstart{V}{isualization} plots are means to visualize data to deduct analytical and statistical conclusions for dat mining and data science.
There are different types of plots used for comparison such as Bar, Lollipop, Bullet, Dot, Range, Radial, Parallel, Radar, Waterfall and etx. or
plots used for correlation analysis such as Heatmap, Bubble, Scatter, Hexagonal and etc. or Pie, Donut, hierarchy and etc, for hierarchical analysis.

In the current report, Box, Scatter, Histogram and Violin plots are examined and implemented by Python to illustrate advantages and disadvantages
of them in a visual and interactive environment with jupyter notebook.


% === II. Box ========================
% =================================================================================
\section{Box}
This mostly used to illustrate the distribution of data, it uses statistical information such as median and quartile to show the density and 
propagation of data in a perfect visualization.

\begin{figure}[!t]
  \centering
  \includegraphics[width=3.5in]{picture/box-cons1.png}
  Box Cons 1.
  \DeclareGraphicsExtensions.
  \caption{Box Cons 1}
  \label{fig:box_cons1}
\end{figure}

\begin{figure}[!t]
  \centering
  \includegraphics[width=3.5in]{picture/box-cons2.png}
  Box Cons 2.
  \DeclareGraphicsExtensions.
  \caption{Box Cons 2}
  \label{fig:box_cons2}
\end{figure}

\subsection*{1.Strength}
\begin{itemize}
  \item Large datasets can be analyzed with box plots. median, upper quartile, lower quartile, min and the max of values can be used for a wide range of analysis and calculations.
  \item Distribution of data can be inspired quickly and simply.
  \item Comparing different types of data can be used get with distribution analysis between multiple box plots.
  \item Outliers can be detected simply by viewing the data outside of the min and max boundaries.
\end{itemize}

\subsection*{1.Weakness}
\begin{itemize}
  \item We can not extract exact values and detailed information from box plots. Box plots are not self explanatory by themselves so they need to be paired with other types of plots such as histograms. according to the following plot, the box plot does not show accurate data to be extracted as values. Fig(\ref{fig:box_cons1})
  \item In the histogram, we can detect that the density of data is high for "VinyISd" between 160000 and 200000, when we look at box plt, the data is shown between 180000 and 220000 and it is because of non gradient view of the box plot. Fig(\ref{fig:box_cons1})
  \item No exact value is shown in the box plots and most of the visible data is related to statistics measures such as median. Fig(\ref{fig:box_cons2})
\end{itemize}

% === III. Histogram ========================
% =================================================================================
\section{Histogram}
Histogram is useful if we want to show the distribution of a continuous data.


\begin{figure}[!t]
  \centering
  \includegraphics[width=3.5in]{picture/histogram-cons1.png}
  Histogram Cons 1.
  \DeclareGraphicsExtensions.
  \caption{Histogram Cons 1}
  \label{fig:histogram_cons1}
\end{figure}

\begin{figure}[!t]
  \centering
  \includegraphics[width=3.5in]{picture/histogram-cons2.png}
  Histogram Cons 2.
  \DeclareGraphicsExtensions.
  \caption{Histogram Cons 2}
  \label{fig:histogram_cons2}
\end{figure}

\begin{figure}[!t]
  \centering
  \includegraphics[width=3.5in]{picture/histogram-cons3.png}
  Histogram Cons 3.
  \DeclareGraphicsExtensions.
  \caption{Histogram Cons 3}
  \label{fig:histogram_cons3}
\end{figure}

\subsection*{1.Strength}
\begin{itemize}
  \item Is acts outstanding in large datasets, this is by using the grouping (bins) as intervals.
  \item We can extract patterns and trends in data, symmetry, skewness and peaks can be detected easily.
  \item Outliers are shown according to the isolated bars showing in a histogram chart.
  \item Quickly get insightful understanding of data.
\end{itemize}

\subsection*{1.Weakness}
\begin{itemize}
  \item Histogram is highly dependent to number of bins, in Cons 1 chart Fig(\ref{fig:histogram_cons1}), with small and large bins, for the bank load duration with a low bin (5) it is not simple to understand max occurrence of a same duration value extraction. in the right side we can not determine if the value is 0 or 1000.
  \item Using histogram for binary data or any data that is not continuous, will not get a good result to distinguish meaningful data. the data is grouped into ranges or intervals, the original data is lost and can price an exact value Fig(\ref{fig:histogram_cons2}). 
  \item In the cons 3, changing the maximum included data makes the chart to be misleading, it is visible in the left and right chart that the distribution can not be detected correctly Fig(\ref{fig:histogram_cons3}).
  \item We can not gather useful information when we compare two different datasets to compare features using histogram, instead we can use bar charts.
  \item Detection of continuos and discrete variables is not possible.
  \item Detection of distribution type is possible but it is so difficult.
  \item All af the data should be in memory, so it is a high-end hardware for large dataset.
\end{itemize}

% === IV. Scatter ========================
% =================================================================================
\section{Scatter}
Histogram is a mathematical graph that uses Cartesian coordinates to show two variables within a data.

\begin{figure}[!t]
  \centering
  \includegraphics[width=3.5in]{picture/scatter-cons1.png}
  Scatter Cons 1.
  \DeclareGraphicsExtensions.
  \caption{Scatter Cons 1}
  \label{fig:scatter_cons1}
\end{figure}

\begin{figure}[!t]
  \centering
  \includegraphics[width=3.5in]{picture/scatter-cons2.png}
  Scatter Cons 2.
  \DeclareGraphicsExtensions.
  \caption{Scatter Cons 2}
  \label{fig:scatter_cons2}
\end{figure}

\subsection*{1.Strength}
\begin{itemize}
  \item Shows trends and relationships between two variables, to detect patterns and correlations in data we can use scattered plots.
  \item Is shows complete data points, illustrates distribution showing the maximum and the minimum of the values.
  \item Correlation detection including positive and negative correlation between variables.
  \item Accurate analysis can be done by scattered plots, this is because of high granularity of the data points.
  \item Outliers can be detected simply with a glance and we can distinguish between harsh outsider data to be assumed as outlier.
\end{itemize}

\subsection*{1.Weakness}
\begin{itemize}
  \item Scatter plot translates data for a subjective analysis, so different people may have different perceptions from the plot and data. In the following example, someone may assume values higher than 450000 as outliers and someone assumes 500000 as outlier. data is shown accurately but the detecting the density of each area of data is perceiving Fig(\ref{fig:scatter_cons1}).
  \item Scatter plot is meant to be used with continuous data. as it is visible we can not inspire any correlation between "year built" and "year sold", also it should not be a relation but because the data is not continuous, we can not get a good estimation of these two variable with scatter plot Fig(\ref{fig:scatter_cons1}).
  \item Scatter plot shows correlation but it necessarily does not mean that there is a cause and effect relationship between two elements. scatter plot shows a relationship between "total basement surface" versus "uniform basement surface" but logically there is no cause and effect relationship between these two variable Fig(\ref{fig:scatter_cons2}).
\end{itemize}

% === V. Violin ========================
% =================================================================================
\section{Violin}
Is a statistical graph to be used for comparison between probability distributions. compared to box plot, it shows more details related to the distribution.

\begin{figure}[!t]
  \centering
  \includegraphics[width=3.5in]{picture/violin-cons1.png}
  Scatter Cons 1.
  \DeclareGraphicsExtensions.
  \caption{Scatter Cons 1}
  \label{fig:violin_cons1}
\end{figure}

\begin{figure}[!t]
  \centering
  \includegraphics[width=3.5in]{picture/violin-cons2.png}
  Scatter Cons 2.
  \DeclareGraphicsExtensions.
  \caption{Scatter Cons 2}
  \label{fig:violin_cons2}
\end{figure}

\subsection*{1.Strength}
\begin{itemize}
  \item Compared to box plots, it shows more comprehensive view of the data's distribution, the peaks, valleys and tails makes viewer to detect the similarity between groups.
  \item To detect unusual clusters of data points by validating the shape and spread of curves, distinct groups can be detected simply.
\end{itemize}

\subsection*{1.Weakness}
\begin{itemize}
  \item Violin plot does not show a good result when there are differences between groups symmetry and skewness and shape. It is not easy to extract accurate information from the plot. For instance in the right plot, we can not detect the number of married, single or divorced. so we need to use violin plots in a combination with other plot such as bar plots or box plots Fig(\ref{fig:violin_cons1}).
  \item Where there is not enough points to be fed into the violin plots, the accuracy of the violin plot drops for the small categories. In the Right side with small data, the plot is so misleading and we can not inspire any meaningful conclusion of it Fig(\ref{fig:violin_cons2}).
\end{itemize}

\ifCLASSOPTIONcaptionsoff
  \newpage
\fi

% trigger a \newpage just before the given reference
% number - used to balance the columns on the last page
% adjust value as needed - may need to be readjusted if
% the document is modified later
%\IEEEtriggeratref{8}
% The "triggered" command can be changed if desired:
%\IEEEtriggercmd{\enlargethispage{-5in}}

% references section

% can use a bibliography generated by BibTeX as a .bbl file
% BibTeX documentation can be easily obtained at:
% http://mirror.ctan.org/biblio/bibtex/contrib/doc/
% The IEEEtran BibTeX style support page is at:
% http://www.michaelshell.org/tex/ieeetran/bibtex/
% \bibliographystyle{IEEEtran}
% argument is your BibTeX string definitions and bibliography database(s)
% \bibliography{IEEEabrv,Bibliography}
%
% <OR> manually copy in the resultant .bbl file
% set second argument of \begin to the number of references
% (used to reserve space for the reference number labels box)
% \begin{thebibliography}{1}

% \bibitem{lin2021ego2hands}
% M.~Valinejad and M.~Y. Daly, \emph{Augmented Reality by Hand Gesture Recognized
% Commands and Movements}, Turkey: Istanbul, 2023.


% \end{thebibliography}

% biography section
% 
% If you have an EPS/PDF photo (graphicx package needed) extra braces are
% needed around the contents of the optional argument to biography to prevent
% the LaTeX parser from getting confused when it sees the complicated
% \includegraphics command within an optional argument. (You could create
% your own custom macro containing the \includegraphics command to make things
% simpler here.)
%\begin{IEEEbiography}[{\includegraphics[width=1in,height=1.25in,clip,keepaspectratio]{mshell}}]{Michael Shell}
% or if you just want to reserve a space for a photo:

% \begin{IEEEbiography}{Michael Shell}
% Biography text here.
% \end{IEEEbiography}

% if you will not have a photo at all:
\begin{IEEEbiographynophoto}{Mehdi Valinejad}
Received the B.S. degree in industrial engineering from the Azad University (South Tehran Branch) in 2012, and is currently working Master's. degree at the University of Bahcesehir at Istanbul.
\end{IEEEbiographynophoto}

% insert where needed to balance the two columns on the last page with
% biographies
%\newpage

% \begin{IEEEbiographynophoto}{Jane Doe}
% Biography text here.
% \end{IEEEbiographynophoto}

% You can push biographies down or up by placing
% a \vfill before or after them. The appropriate
% use of \vfill depends on what kind of text is
% on the last page and whether or not the columns
% are being equalized.

%\vfill

% Can be used to pull up biographies so that the bottom of the last one
% is flush with the other column.
%\enlargethispage{-5in}



% that's all folks
\end{document}


